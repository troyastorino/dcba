\documentclass[10pt]{article}

% small margins
\usepackage[margin=1in]{geometry}

% no large spaces in lists
\usepackage{enumitem}
\setlist{nolistsep}

\title{\vspace{-4em}6.S078 Update}
\author{Troy Astorino \and Turner Bohlen \and Craig Cheney \and Gus Downs}
\date{April 2, 2013}

\begin{document}
\maketitle
\vspace{-4em}

\section{Plan Progress}
We've had exciting progress this week!  We applied to the Summer@Highland
incubator on Monday, and are applying to the MIT Founders' Skills Accelerator on
Friday. 

In terms of progress, we had a lot of work on the software side completed over
spring break.  We have
also moved farther along in eliminating potential low-cost projection
technologies. These points will be covered in more depth in the prototype
progress section.

\section{Prototype Progress}
The software advanced a large amount since the last update. The system now
reconstructs a point cloud from a series of images taken with a DLP projector!
The particular scan used was based on gray code scanning, an approach we are not
planning on using (because it does require a DLP Projector), but was a good
testable starting point (and 
it is very nice to finally have a scan of something).  The system was written
with easy adaption to other scanning techniques in mind.

In terms of projection technologies, we have eliminated Moire Contours as a
technique that can be used with structured light scanning.  If we were to use
Talbot imaging, we found we would require wide beam laser light.  This is also
the case if we were to effectively use optical interference.  Another option
incorporating lasers would be to 'paint' laser lines on the object while
the camera aperture is open. Simple projection through a grating is still a possibility.

A couple pinhole cameras we ordered arrived, and we have started playing with those.

\section{Baffling Variables}
The largest unknown remains execution, and in particular the ability to
successfully incorporate a projection technology that is cheaper than DLP
Projection. We're putting the majority of our effort into that.

\section{Seven Day Plan}
\begin{itemize}
\item Finish the accelerator applications successfully.
\item Continue software development, incorporating meshing and potentially
  adding a swept-plane scanning technique (again, not a
  technique we will ultimately use, but good to ensure the flexibility of the
  code)
\item Put the pinhole cameras on the Arduino boards and start capturing images
\item Order gratings and LEDs powerful enough to test our hypotheses about
  LED+grate scanning
\end{itemize}

\section{People to Meet}
We are meeting with Ben Einstein, the director of Bolt, to discuss whether we
would be a good fit for the Bolt hardware incubator. 

\section{Desired Resources}
We are very thankful to Professor Gifford for continuing to probe the course's
VC friends about prototyping funding.

\end{document}