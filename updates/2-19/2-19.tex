\documentclass[10pt]{article}

\usepackage[margin=1in]{geometry}

\usepackage{enumitem}
\setlist{nolistsep}

\title{\vspace{-4em}6.S078 Update} \author{Troy Astorino \and Turner Bohlen \and
Craig Cheney \and Gus Downs}

\begin{document}
\maketitle
\vspace{-4em}

\section{Plan Progress} After meeting with Professor Gifford last week, we were
left with 3D scanning and LIDAR as our two potential products. We decided to
pursue 3D scanning, thinking that there are larger potential markets for 3D
scanning, and that the core technology can be applied to different market
segments. The next week will be a combination of research into existing 3D
scanning technologies, early prototype sketches, and research into potential 3D
scanning markets.

Prior to this week's research, our guess is that low-cost 3D scanner aimed at consumers, riding
alongside the current excitement in consumer 3D printing, could be viable. That
target customer would be a digital fabrication hobbyist/enthusiast/early-adopter,
or even a small-scale manufacturer (machine shops). Our design ideas would be to
bring resolution smaller than typical machining tolerances (0.1 mm) while
keeping the unit affordable (below \$500). However, we are exicited to see what
other, potentially more ambitious, applications of the technology come up in
this week's research.

This week's plan is to refine our market analysis by looking into a number of
particular potential applications of 3D scanning, as well as to research the
existing 3D scanning methods. By the end of this week ( +2/22) we would like to
have selected a few preliminary design schemes so that we can start prototyping.

Potential markets for further research this week:
\begin{itemize}
\item Hobbyist and small manufacturing/prototyping
\item Hobbyist art and 3D scanning for the sale of artwork on sites such as Etsy
\item Restoration and documentation of 3D artwork, primarily sculptures
\item Cheap, regular, full-body scanning for medical applications - scoliosis, dermetology, etc.
\item Interior design and architecture - cheap full-interior scans of rooms (see
  matterport)
\end{itemize}

Possible technologies:
\begin{itemize}
\item Stereo cameras
\item Camera array
\item Stereo or single camera paired with laser triangulation
\item Laser triangulation
\item Structured light
\item Laser time-of-flight
\item Variable focal distance scanning
\end{itemize}

\section{Prototype Progress} As we have yet to select what approach we are
taking to 3D scanning, we have not started building prototype software or
hardware. We plan to start making progress on a prototype by the end of this
week.

\section{Baffling Variables}
This week will reveal more potential difficulties for the idea. The above goals for this week - researching technologies and markets - may be the two components of this project that could be best described as baffling at the moment.

\newpage

\section{Seven Day Plan} 
\begin{itemize} 
\item \textbf{Team goals}
\begin{itemize}
\item Each team member will research 3 existing 3D printing technologies - estimated difficulty,
    expense per unit, and time to prototype for each
\end{itemize}

\item \textbf{Troy}
\begin{itemize}
\item Contact 6.838 (Advanced topics in computer graphics: Computational Fabrication) professor and run
  through their 3D scanning assignment with Turner
\item Obtain funding for prototyping
\end{itemize}

\item \textbf{Craig}
\begin{itemize}
\item Contact ex-employer (AutoDesk) to research
  what is required for 3D scanning in consumer/industrial CAD. 
\end{itemize}

\item \textbf{Gus}
\begin{itemize}
\item contact MITERS, the hobby shop, Artisans Asylum, BOLT, LEMNOS labs, and
  Tech Shop to research the hobbyist space
\item Contact early purchasers of 3D printers (possibly through MakerBot or
  open-source 3D printing forums) to gauge hobbyist
  interest/requirements/use cases for a 3D scanner
\end{itemize}

\item \textbf{Turner}
\begin{itemize}
\item Work through 6.838 3D scanning assignment with Troy
\item Contact friend working with Form Labs and venture capitalist friends to
  better understand the applications of 3D scanning they see, and so inform our
  own market decision
\end{itemize}
\end{itemize}

\section{People to Meet} We haven't identified any particular people we would
like to meet at this stage (other than the professor for 6.838 listed
above...we'll let you know if we have difficulty getting through to him). We
would like to have discussions with a number of market experts, such as 3D
printing, art, hobbyist hardware, medical, and architectural technology experts
the possible applications within each market and potential market value. One way
to approach this is to talk with venture capitalists in these areas who every
day see multiple potential new technologies and so may be able to give us
valuable feedback.

\section{Desired Resources} In order to start prototyping we need some capital
to purchase components. We have access to all of the facilities that we will
need, but at present we lack the funding to purchase materials. In our meeting
with Professor Gifford he mentioned that it would be possible to get some form
of funding, such as a forgivable loan, from the course's VC partners. We would
like to pursue that possibility.

\end{document}