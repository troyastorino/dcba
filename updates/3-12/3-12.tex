\documentclass[10pt]{article}

% small margins
\usepackage[margin=1in]{geometry}

% no large spaces in lists
\usepackage{enumitem}
\setlist{nolistsep}

\title{\vspace{-4em}6.S078 Update}
\author{Troy Astorino \and Turner Bohlen \and Craig Cheney \and Gus Downs}

\usepackage{hyperref}

\begin{document}
\maketitle
\vspace{-4em}

\section{Plan Progress}
This week has been focused on prototype progress, as weeks will most likely be
for the rest of the class.  In our meeting with Dr. Fantone (an optics
consultant who has done consumer, medical, and military optical metrology work),
e clarified our understanding of how our implementation concepts would work, and
gained knowledge of some techniques that should prove to be very helpful. Dr. Fantone
also expressed that he thought our accuracy goals (0.1 mm) are very
conservative, even 'sloppy', which gives additional hope that our low price
point is achievable.  

We also noticed a few potential competitors emerge in the market.  At SXSW
MakerBot announced the Digitizer, a desktop 3D scanner they are developing.  It
is still in prototype phase, so there isn't any information available on price
and accuracy.  Based on the price of MakerBot's 3D printers and their
description of the technology they were using, we expect the Digitizer will come
in at a higher price point than we are aiming for, but time will tell.
Additionally, an Kickstarter campaign for a company called CADScan is very near
its \pounds 80,000 funding goal with 6 days left; we expect it will get funded.
CADScan is also producing a desktop 3D scanner. Based off the Kickstarter
campaign they are planning on a price at least over \$1,000. There wasn't
explicit information on accuracy. These competitors show that we aren't
completely crazy in thinking there could be a large market here, but it does
mean that we are going to have to be more aware of the competitive landscape.

We also discussed what the team's plans for continuation of the project after
the class. We have two members that are planning to continue working on it over
the summer, and will be applying to various accelerators for that.  We also
discussed what commitments could be after that point, and came to some
conclusions. Obviously, these conclusions would be flexible enough to adapt to
changing circumstances.

\section{Prototype Progress}
We've started software development in earnest. We've designed our system so we
can start coding for use with USB Webcams and external display projectors, and
will be able to (relatively) easily transition the code to our final system. If
you want to be able to look at our software development progress, please email us
(\href{mailto:dcba@mit.edu}{dcba@mit.edu}) and we can add you to our private GitHub repository.

We ordered initial parts for hardware prototyping as well. We've laid out a
reasonable development plan that has us going end-to-end on our test hardware,
from imaging to 3D model, by the 8th of March.

Our meeting with Dr. Fantone really helped us elucidate the plan for projecting
light on the object. We came to him with the projection options we had
previously brainstormed, and of those only having a grating in front of an LED
really survived. The name for the grating projection technique we would use is
Talbot imaging.  The other concepts we decided to look into after this meeting
are creating patterns interferometrically (i.e. optical fringes) and Moire
contouring. 

\section{Baffling Variables}
None are particularly baffling at the moment. The biggest unknown for us is execution.

\section{Seven Day Plan}
This is less of a seven day plan as much as a near future development plan:

The software plan has us generating a triangulated point cloud by Saturday 3/16, and
using existing graphical software to mesh that into a surface by next Tuesday, 3/19. We
then have the meshing process automated (using the libraries for the existing
software) by next Friday, 3/22.

Over the next week, hardware has two separate tasks. First, it is integrating the sensors and chips we
ordered into a microcontroller which will be able to talk with the software
program.  Second, it needs to decide on what we need to test in order to decide
between our different projection options. It also need to start ordering the
 parts necessary for these tests.

\section{People to Meet}
We are doing fine on this front.

\section{Desired Resources}
Also doing OK. We're still working on getting prototyping funding, and am
picking up that conversation with Professor Gifford again on 3/17.

\end{document}